\pagestyle{empty}

\hfill

\vfill


\section*{Colofón}

%This document was typeset, using the typographical  \texttt{classicthesis} look--and--feel, in \LaTeX.

\textit{Figura de cubierta: El decaimiento de una partícula lambda en la camara de burbujas de hidrógeno de 32 cm} \\
Esta imagen de 1960  recoge trazas una colisión de partículas real en la primera cámara de burbujas del CERN usada en experimentos. Era un detector pequeño, para los estándares actuales, con solo $32$ cm de diámetro. Los piones, negativamente cargados con una energía de $16$ GeV, entraban por la izquierda. 
Uno de ellos interaccionaba con un protón en el hidrógeno líquido y creaba trazas de nuevas partículas, incluyendo a una partícula neutra (una lambda), que decae a dos partículas cargadas que tienen forma de V, en el centro. Las partículas cargadas de menor energía producen espirales debido al campo magnético de la cámara.
La invención en 1952 de la cámara de burbujas revoluciona el campo de la física de partículas, permitiendo ver y fotografiar las trazas de las partículas, después de liberar la presión que mantenía el líquido por encima de su punto de ebullición normal \cite{CERN-EX-11465}.\\
\textit{La imagen está levemente modificada, pues fue sometida a un proceso de vectorización.}

\bigskip

\noindent Todas las figuras no referenciadas son creadas por el autos con fines ilustrativos, pudiendo dichas figuras estar basadas en otras similares. Pueden usarse dichas figuras bajo la licencia \href{https://www.gnu.org/licenses/gpl.html}{GPLv3}  descargándose desde \href{https://github.com/marromlam/thesis-pictures.git}{GitHub}.

%All the non-referenced figures are created and made properly by the author, based on previous similar pictures. They can be used under de \href{https://www.gnu.org/licenses/gpl.html}{GPLv3} and downloaded from \href{https://github.com/marromlam/thesis-pictures.git}{GitHub}.

\bigskip

\noindent\begin{minipage}{0.75\textwidth}
\noindent\finalVersionString
\end{minipage}
\normalsize