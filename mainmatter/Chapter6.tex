%%%%%%%%%%%%%%%%%%%%%%%%%%%%%%%%%%%%%%%%%%%%%%%%%%%%%%%%%%%%%%%%%%%%%%%%%%%%%%%%
%%%%%%%%%%%%%%%%%%%%%%%%%%%%%%%%%%%%%%%%%%%%%%%%%%%%%%%%%%%%%%%%%%%%%%%%%%%%%%%%
%%%%%%%%%%%%%%%%%%%%%%%%%%%%%%%%%%%%%%%%%%%%%%%%%%%%%%%%%%%%%%%%%%%%%%%%%%%%%%%%
\chapter{Conclusiones}
%%%%%%%%%%%%%%%%%%%%%%%%%%%%%%%%%%%%%%%%%%%%%%%%%%%%%%%%%%%%%%%%%%%%%%%%%%%%%%%%
%%%%%%%%%%%%%%%%%%%%%%%%%%%%%%%%%%%%%%%%%%%%%%%%%%%%%%%%%%%%%%%%%%%%%%%%%%%%%%%%
%%%%%%%%%%%%%%%%%%%%%%%%%%%%%%%%%%%%%%%%%%%%%%%%%%%%%%%%%%%%%%%%%%%%%%%%%%%%%%%%


Se ha presentado un marco de trabajo basado en la paralelización en GPU para realizar un análisis de la distribución angular de los productos de desintegración del canal $\Bs \rightarrow \Jpsi \kaon \antikaon$. En general, el código está escrito centrándose en la región de masa del $\fai(1020)$. 

En la parte de simulación, se terminó el desarrollo de un código para \textsc{EvtGen}, y se corrigieron los errores que este presentaba. Dicho código, permitirá generar simulaciones MC completas incluyendo efectos del detector para el análisis completo de los datos del Run II (2015--2018). Como se mencionaba en secciones previas, el programa contiene una descomposición en amplitudes de polarización que incluye onda S no--resonante y resonante, onda P y onda D, lo que ayudará también a análisis $\Bs \rightarrow \Jpsi \kaon \antikaon$ en regiones altas de masa, donde predomina la onda D.
Siguiendo esta línea, también se escribió un código para GPU que genera \emph{toy} MC, y que permite generar las distribuciones angulares con el mismo código que se ajustan los datos reales. Este código ha permitido la validación de incertidumbres sistemáticas tanto del bias intrínseco del ajuste como de los factores $C_{\text{SP}}$.

Por otra parte, se muestran los resultados obtenidos con otros dos códigos, ambos por máxima verosimilitud y escritos para GPU: el ajuste a la aceptancia en el tiempo de desintegración y el ajuste a la distribución angular. Con el primero de ellos se logra mejorar el tiempo de cálculo del código que se viene usando hasta ahora, lo que en un futuro con mayor cantidad de datos, cobrará gran valor. Con el segundo se añade la funcionalidad de ajustar también onda D, generalizando por tanto el código para una más amplia región de masa.

 
En líneas generales, se sienta la base para hacer un futuro análisis completo del Run II de \lhcb (años 2015 a 2018), que resultará en la más precisa media de los parámetros de violación CP del $\Bs$.