%%%%%%%%%%%%%%%%%%%%%%%%%%%%%%%%%%%%%%%%%%%%%%%%%%%%%%%%%%%%%%%%%%%%%%%%%%%%%%%%
%%%%%%%%%%%%%%%%%%%%%%%%%%%%%%%%%%%%%%%%%%%%%%%%%%%%%%%%%%%%%%%%%%%%%%%%%%%%%%%%
%%%%%%%%%%%%%%%%%%%%%%%%%%%%%%%%%%%%%%%%%%%%%%%%%%%%%%%%%%%%%%%%%%%%%%%%%%%%%%%%
\chapter{Introducción}
\label{cha:intro}
%%%%%%%%%%%%%%%%%%%%%%%%%%%%%%%%%%%%%%%%%%%%%%%%%%%%%%%%%%%%%%%%%%%%%%%%%%%%%%%%
%%%%%%%%%%%%%%%%%%%%%%%%%%%%%%%%%%%%%%%%%%%%%%%%%%%%%%%%%%%%%%%%%%%%%%%%%%%%%%%%
%%%%%%%%%%%%%%%%%%%%%%%%%%%%%%%%%%%%%%%%%%%%%%%%%%%%%%%%%%%%%%%%%%%%%%%%%%%%%%%%


El estudio de la violación de la simetría carga--paridad (CP), tiene un enorme interés científico. Las simetrías constituyen la base de nuestro entendimiento de la naturaleza que nos rodea. En virtud del teorema de Noether se sabe que toda simetría se corresponde con una ley de conservación, unas de las más básicas leyes que fundamentan nuestro pensamiento. La comprensión plena de estas ideas sencillas, elegantes y profundas nos permite ordenar nuestro saber y conocer los fundamentos que rigen este Universo \cite{romeroTFG}.

La violación de la simetría CP es una de las partes de la Física que viene siendo estudiada en gran profundidad en las últimas décadas. Experimentos completos fueron construidos para esclarecer la naturaleza de esta simetría, que presenta grandes incógnitas tanto en la parte teórica como en la experimental, como lo son Belle, BaBar y el propio \lhcb.
%%
El Modelo Estándar (SM) de la Física de Partículas, describe las interacciones fundamentales entre las partículas y desde su formulación entre los años 60 y 70, demostró tener un enorme poder predictivo en medidas de mucha precisión. Sin embargo, sí existen observaciones experimentales en la que ha fracasado: la oscilación de neutrinos \cite{maltoni2004status} implica que estos sean masivos, lo que contradice el \stdmod; observaciones astronómicas indican que existe una contribución a la masa del Universo de entre 5 y 6 veces más grande que la materia común, la materia oscura (DM) \cite{bertone2005particle}, y que solo se observa mediante fenómenos gravitacionales; o por supuesto la asimetría entre materia y antimateria que se observa en el Universo. Es de hecho esto último lo que motiva los estudios de violación CP, puesto que es, junto con la violación del número bariónico y las interacciones fuera del equilibrio, uno de los ingredientes de las condiciones de Sájarov \cite{Sakharov:1967dj} para satisfacer el ratio materia--antimateria que hay en el Universo.


En el terreno teórico existen también grandes motivaciones para indagar en la violación CP. 
Primeramente, la mayor parte de los parámetros de los que requiere el \stdmod provienen de término de Yukawa (\emph{vid.} \S\ref{sec_yukawa}), lo que hace que el \stdmod parezca una teoría efectiva de baja energía. El \stdmod tampoco ofrece una explicación para que existan 3 familias de fermiones, es una entrada que ha de imponérsele al modelo. Además, probablemente la laguna más conocida, no existe una teoría cuántica de la gravedad, por lo que a energías del orden de la escala de Planck, $E_\mathrm{P} = \sqrt{\frac{\hbar c^5}{G}} =  1.9561\times10^{9} \, \mathrm{J} = 1.22104\times10^{28} \, \mathrm{eV}$, el \stdmod es inválido.


La existencia de nuevos fenómenos no predichos por el \stdmod, es decir, física más allá del \stdmod (\bstdmod), podría introducir efectos visibles en los observables de violación CP. En el \stdmod la violación CP se describe con la teoría de Cabbibo--Kobayashi--Maskawa, que mediante una fase compleja es capaz de mezclar estados de masa de partículas.
Gracias al fenómeno de mezcla, los mesones neutros $\mathrm{B}$ ofrecen multitud de canales de desintegración para hacer tests de precisión del  \stdmod.
Concretamente, en el decaimiento del $\Bs$, la violación CP puede emanar de la interacción entre el decaimiento de $\Bs$ y el $\Bbs$  precedida de una oscilación entre ambos, y se manifiesta a través de un valor no nulo de $\phis = \text{arg}(\lambda)$ ---muy pequeña en el \stdmod)--- siendo $\lambda$ el parámetro que describe la violación CP en la interferencia entre la mezcla y la desintegración.
Muchos modelos de física \bstdmod predicen valores más grandes de esta fase, aún cumpliendo las demás constricciones; por lo que medir un valor significativamente distinto de la predicción del \stdmod indicaría una clara evidencia de procesos \bstdmod.


En esta tesitura, el canal más sensible para ver contribuciones de física \bstdmod es el $\Bs \rightarrow \Jpsi(\rightarrow \muon \antimuon) \kaon \antikaon$, donde la mayor parte de los kaones del estado final provienen de la resonancia intermedia $\fai(1020)$ \cite{faller2009precision}. Para acceder a los parámetros observables de la oscilación y desintegración del mesón será necesario desentrelazar las distintas componentes pares e impares bajo el operador $\OPcp$ (que invierte la carga y paridad de una partícula), para lo que será requerimiento hacer un estudio de la distribución angular de muones y kaones midiendo: $\varphi_{\text{s}}$, la fase electrodébil ya mencionada;  $\Delta\Gamma_{\text{s}}$, la diferencia de las anchuras entre los autoestados de $\OPcp$; $\Gamma_{\text{s}}-\Gamma_{\text{d}}$, la diferencia entre la anchura del $\Bs$ y del $\text{B}_{\text{d}}$ (canal de control); y $\Delta m$, a diferencia de masa entre los autoestados de $\OPcp$.


Este Trabajo Fin de Máster muestra un anticipo de una Tesis Doctoral que tratará de obtener los parámetros de oscilación del $\Bs$ con la mayor precisión, valiéndose para ello de toda la muestra de datos tomada por \lhcb.
El trabajo se organiza en 6 capítulos en los que se desgrana el proceder para hacer un test al \stdmod que involucra la determinación de $\varphi_{\text{s}}$. 
El Capítulo \ref{cha:theo} resume brevemente los temas más relevantes del \stdmod vinculados a la fase electrodébil $\varphi_s$. 
Seguidamente, en el Capítulo \ref{cha:pheno}, se describe la fenomenología del canal estudiado $\Bs \rightarrow \Jpsi \kaon \antikaon$, mostrándose cómo a partir de los datos se puede extraer la fase.
El Capítulo \ref{cha:detector} versa sobre las características de \lhcb, desde su situación general hasta una vista breve de sus detectores, que permiten a fin de cuentas medir las partículas de la desintegración.
A continuación, el Capitulo \ref{cha:tools} describe las técnicas empleadas para hacer el tratamiento de los datos, así como el método empleado.
Por último el Capítulo \ref{cha:ana} muestra los resultados y la forma en cómo se procede a su obtención, así como también las conclusiones. 
%
%Además, cada uno de los capítulos presenta los apéndices precisos para complementar el texto principal, sin así entorpecer la comprensión de la línea fundamental.


%%%%%%%%%%%%%%%%%%%%%%%%%%%%%%%%%%%%%%%%%%%%%%%%%%%%%%%%%%%%%%%%%%%%%%%%%%%%%%%%
%%%%%%%%%%%%%%%%%%%%%%%%%%%%%%%%%%%%%%%%%%%%%%%%%%%%%%%%%%%%%%%%%%%%%%%%%%%%%%%%
%%%%%%%%%%%%%%%%%%%%%%%%%%%%%%%%%%%%%%%%%%%%%%%%%%%%%%%%%%%%%%%%%%%%%%%%%%%%%%%%